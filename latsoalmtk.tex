\documentclass{report}
\usepackage{graphicx}
\usepackage[Export]{adjustbox}
\usepackage{hyperref}
\usepackage{enumitem}
\usepackage{amsmath}
\usepackage[makeroom]{cancel}
\usepackage{xcolor}
\usepackage{titlesec}
\usepackage{chngcntr}
\usepackage{accents}

\hypersetup{
    colorlinks=true,
    linkcolor=blue,
    filecolor=magenta,      
    urlcolor=cyan,
    pdftitle={Latihan Soal PSAJ Matematika},
    pdfpagemode=FullScreen,
    }
    
\urlstyle{same}

\newcommand{\options}[5]{
\begin{enumerate}[label=\alph*.]
	\item #1
	\item #2
	\item #3
	\item #4
	\item #5
\end{enumerate}
}

\newcommand{\pemb}{ \textbf{Pembahasan} \\}

\titleformat{\chapter}[display]
  {\normalfont\bfseries}{}{0pt}{\Large}

\counterwithin*{equation}{section}
\counterwithin*{equation}{chapter}

\begin{document}

\chapter{Eksponen}

\section*{Sifat Eksponen}

\begin{equation}
\label{ex_mult}
a^{n} \cdot a^{m} = a^{n+m} 
\end{equation}

\begin{equation}
\label{ex_div}
\frac{a^{n}}{a^{m}} = a^{n-m} 
\end{equation}

\begin{equation}
\label{ex_pow}
(a^n)^m = a^{nm}
\end{equation}

\begin{equation}
\label{ex_neg}
a^{-n}=\frac{1}{a^n}
\end{equation}

\begin{equation}
\label{ex_frac}
a^{\frac{n}{m}}=\sqrt[n]{a^m}
\end{equation}

\begin{equation}
\label{ex_div_pow}
\left(\frac{a}{b}\right)^{n} = \frac{a^{n}}{b^{n}}
\end{equation}

\begin{enumerate}
% No. 1
\item
Bentuk sederhana dari $\left(\frac{x^{2}y^{3}z^{-1}}{x^{-3}y^{-4}z^{3}}\right)^3$ adalah \ldots
\options
{$\frac{x^{15} y^{21}}{z^6}$}
{$\frac{x^{15} y^{21}}{z^8}$}
{\textcolor{red}{{$\frac{x^{15} y^{21}}{z^{12}}$}}}
{$\frac{x^{15} y^{21}}{z^{10}}$}
{$\frac{x^{15} y^{21}}{z^{14}}$}

\pemb

Menggunakan sifat eksponen \ref{ex_pow}
\[
	\left(\frac{x^{2}y^{3}z^{-1}}{x^{-3}y^{-4}z^{3}}\right)^3 = \frac{x^{6}y^{9}z^{-3}}{x^{-9}y^{-12}z^{9}}
\]

Menggunakan sifat eksponen \ref{ex_neg}
\[
	\frac{x^{6}y^{9}z^{-3}}{x^{-9}y^{-12}z^{9}} =
	\frac{x^{6}y^{9}x^{9}y^{12}}{z^{3}z^{9}}
\]

Menggunakan sifat eksponen \ref{ex_mult}
\[
	\frac{x^{6}y^{9}x^{9}y^{12}}{z^{3}z^{9}} = 
	\frac{x^{15} y^{21}}{z^{12}} \text{(C)}
\]


% No. 2
\item
Nilai dari $\left(\frac{1}{8}\right)^{\frac{-2}{3}}+32^{\frac{2}{5}}+27^{\frac{2}{3}}$ adalah \ldots
\options
{8}
{9}
{10}
{14}
{\textcolor{red}{17}}

\pemb

Menggunakan sifat eksponen \ref{ex_pow}
\[
	\left(\frac{1}{8}\right)^{\frac{-2}{3}}+32^{\frac{2}{5}}+27^{\frac{2}{3}} = 
	\sqrt[3]{\left(\frac{1}{8}\right)^{-2}}+\sqrt[5]{32^{2}}+\sqrt[3]{27^{2}}
\]

Menggunakan sifat eksponen \ref{ex_div_pow}
\[
	\sqrt[3]{\left(\frac{1}{8}\right)^{-2}}+\sqrt[5]{32^{2}}+\sqrt[3]{27^{2}} =
	\sqrt[3]{\frac{1^{-2}}{8^{-2}}}+\sqrt[5]{32^{2}}+\sqrt[3]{27^{2}}	
\]

Menggunakan sifat eksponen \ref{ex_neg} \\
\begin{align*}
								\sqrt[3]{\frac{1^{-2}}{8^{-2}}}+\sqrt[5]{32^{2}}+\sqrt[3]{27^{2}} 
								&= \sqrt[3]{\frac{\frac{1}{1^{2}}}{\frac{1}{8^{2}}}}+\sqrt[5]{32^{2}}+\sqrt[3]{27^{2}}\\
								&= \sqrt[3]{64}+\sqrt[5]{32^{2}}+\sqrt[3]{27^{2}}\\
								&= 4 + 2^2 + 3^2 \\
								&= 4 + 4 + 9 \\
								&= 17  \text{(E)}
\end{align*}

% No. 3
\item
Hasil dari $2\sqrt{48}-4\sqrt{75}+3\sqrt{12}$ = \ldots
\options
{$18\sqrt{3}$}
{$12\sqrt{3}$}
{$3\sqrt{3}$}
{$-3\sqrt{3}$}
{\textcolor{red}{$-6\sqrt{3}$}}
\pemb
Karena semua pilihan memiliki $\sqrt{3}$, maka akar-akar disederhanakan ke bentuk $x\sqrt{3}$
\begin{align*}
	2\sqrt{48}-4\sqrt{75}+3\sqrt{12} 
	&= 2\sqrt{16.3}-4\sqrt{25.3}+3\sqrt{4.3} \\
	&= 2.4\sqrt{3}-4.5\sqrt{3}+3.2\sqrt{3} \\
	&= 8\sqrt{3}-20\sqrt{3}+6\sqrt{3} \\
	&= -6\sqrt{3} \text{(E)}
\end{align*}

% No. 4
\item
Bentuk sederhana dari $\frac{3\sqrt{5}}{3\sqrt{5}+\sqrt{3}}$ adalah \ldots
\options
{$\frac{15-\sqrt{15}}{-14}$}
{$\frac{15+\sqrt{15}}{-14}$}
{$\frac{15-\sqrt{15}}{14}$}
{$\frac{12-\sqrt{15}}{-14}$}
{$\frac{12-\sqrt{15}}{15}$}
\pemb
Rasionalkan akar dengan cara mengalikan pembilang dan penyebut dengan pasangan sekawan dari penyebut
\begin{align*}
	\frac{3\sqrt{5}}{3\sqrt{5}+\sqrt{3}} \cdot \frac{3\sqrt{5}-3\sqrt{3}}{3\sqrt{5}-\sqrt{3}}
	&= \frac{3\sqrt{5}\left(3\sqrt{5}-3\sqrt{3}\right)}{45-3} \\
	&= \frac{\cancel{3}\sqrt{5}\left(3\sqrt{5}-3\sqrt{3}\right)}{\cancelto{14}{42}} \\
	&= \frac{\sqrt{5}\left(3\sqrt{5}-3\sqrt{3}\right)}{14} \\
	&= \frac{3.5-3\sqrt{3}}{14} \\
	&= \frac{15-3\sqrt{3}}{14}
\end{align*}

\chapter{Logaritma}
\section*{Sifat logaritma}

\begin{equation}
\label{log_chain}
{}^a\log_{b} . {}^b\log_{c} . {}^c\log_{d} = {}^a\log_{d}
\end{equation}

\begin{equation}
\label{log_ext}
{}^{{a}^{m}}\log_{b^{n}} = \frac{n}{m}{}^a\log_{ b}
\end{equation}

% No. 5
Nilai dari ${}^{16}\log_{81}.{}^{3}\log_{125}.{}^{5}\log_{32}$ adalah \ldots
\options
{8}
{10}
{15}
{28}
{32}
\pemb

\begin{align*}
	{}^{16}\log_{81}.{}^{3}\log_{125}.{}^{5}\log_{32} = {}^{2^{4}}\log_{3^{4}}.{}^{3^1}\log_{5^3}.{}^{5^1}\log_{2^5}
\end{align*}
Menggunakan sifat logaritma \ref{log_ext}
\begin{align*}
	{}^{2^{4}}\log_{3^{4}}.{}^{3^1}\log_{5^3}.{}^{5^1}\log_{2^5} = \frac{4}{4}\cdot\frac{3}{1}\cdot\frac{5}{1}{}^{2}\log_{3}.{}^{3}\log_{5}.{}^{5}\log_{2}
\end{align*}
Menggunakan sifat logaritman \ref{log_chain}
\begin{align*}
	15{}^{2}\log_{3}.{}^{3}\log_{5}.{}^{5}\log_{2} 
	&= 15 {}^{2}\log_{2} \\
	&= 15
\end{align*}


\end{enumerate}
\end{document}